\documentclass[DM,lsstdraft,toc]{lsstdoc}
\input{meta}

\setDocChangeRecord{%
  \addtohist{\vcsRevision}{\vcsDate}{Unreleased draft.}{J.D.~Swinbank}
}

\title[DM QA WG]{Data Management QA Strategy Working Group Charge}

\author{John D. Swinbank}
\setDocRef{\lsstDocType-\lsstDocNum}
\setDocDate{\vcsDate}
\setDocUpstreamLocation{\url{https://github.com/lsst/\lsstDocType-\lsstDocNum}}

\setDocAbstract{This is the charge for Data Management QA Strategy Working
Group, to be convened in April 2018.}

\begin{document}
\maketitle

\section{Scope}

This working group (WG) is charged with defining the plan for ``quality
assurance'' (QA) within the Data Management Subsystem.

For the purposes of this WG, we identify the scope of QA as the following:

\begin{itemize}

  \item{Providing developers and commissioners of the LSST Science Pipelines
  with tooling and datasets which will enable them to identify, understand and
  resolve --- or avoid altogether, where possible --- algorithmic issues or
  pathologies.}

  \item{Providing mechanisms by which the Science Pipelines are verified to
  run on Data Facility-provided hardware at a scale appropriate to demonstrate
  their readiness for operations.}

  \item{Tracking progress relative to numerical algorithmic performance
  targets as defined in \citeds{LSE-61} or other DM requirements
  documents\footnote{We deliberately avoid the term ``key performance
  metric'', since existing KPMs (described in \citeds {LDM-502}) are
  ill-defined, presupposing as they do the existence of a complete LSST
  system before it is available.}.}

  \item{Tracking computational performance and enabling the rapid
  identification of performance regressions.}

\end{itemize}

Explicitly excluded from this scope are:

\begin{itemize}

  \item{General purpose ``science validation'' activities, which will be
  separately coordinated by the DM Subsystem Scientist.}

  \item{Usability or other general-purpose improvements to the codebase.}

  \item{Testing of other parts of the DM Subsystem deliverables, including but
  not limited to Data Facility systems, large scale databases, or the Science
  Platform, except in so far as their availability may be necessary to carry
  out QA tasks on the Science Pipelines.}

\end{itemize}

Note that this WG is not charged with delivering tools explicitly designed for
use by the Commissioning Team. It is anticipated that their needs overlap in
large part with those of DM developers, but a separate requirements gathering
exercise, likely conducted in conjunction with the DM Subsystem
Scientist\footnote{And/or DM Validation Scientist, as and when one is
appointed}, may be necessary to address Commissioning. However, the DM
representative to the Commissioning Team, Simon Krughoff, is expected to serve
on this WG (see \S\ref{sec:members}) to ensure that the overall direction of
travel is aligned with the expectations of commissioning.

\section{Period}
\label{sec:period}

The working group will convene before the end of April 2018. Its remit will
expire, and deliverables must be provided, by the end of June 2018.

\section{Responsibilities}

This WG has the following responsibilities:

\begin{itemize}

  \item{Collect input from stakeholders, including the DM System Science Team,
  Science Pipelines Leadership Team, DM Developers, and DM Systems Engineering
  Team, to develop a collection of QA use cases.}

  \item{Map those use cases to existing QA tools or procedures within DM,
  where possible, and identify where new tools or procedures need to be
  developed.}

  \item{Develop requirements documentation covering all tools or procedures,
  new or existing, identified above.}

  \item{Produce a list of datasets which should be curated in support of QA
  activities, and develop a strategy for the management of those
  datasets\footnote{Where possible, datasets should be as specific as
  possible, but the WG may also suggest certain \textit{types} of dataset
  which should be compiled after the WG has been completed.}.}

  \item{Propose a development plan for delivering the required tools,
  procedures and datasets for consideration by DM Project Management.}

\end{itemize}

\section{Specific Considerations}

Topics for discussion by the WG should include, but are not limited to:

\begin{itemize}

  \item{To what extent can all interactive QA use cases be captured within the
  framework of the Science Platform? On what timescale will that be available
  in a form that is useful to Pipelines developers?}

  \item{At what granularity and cadence are integration tests required? Can
  these be scheduled automatically, e.g. by Jenkins, or do they require manual
  intervention for large scale tests?}

  \item{Current thinking has Firefly as LSST's primary image visualization
  tool during operations and (presumably) commissioning, but direct support of
  Pipelines developers is outside its scope. Do we need additional tooling
  here? Is Firefly's existing development plan adequate to DM's needs? Must
  the scope of Firefly development be extended?}

  \item{What tools are required for ad hoc plotting? Should Bokeh, Holoviews
  or other Python packages be formally adopted as part of the LSST software
  stack? Are LSST-specific interfaces required?}

  \item{What is the form of metrics that will be captured by the SQuaSH system
  \citeds{SQR-009}?}

  \item{How should pipelines be instrumented to supply those metrics (see e.g.
  \citeds{DMTN-057})?}

  \item{By what mechanism is a regularly updated pipeline output dataset made
  available for test purposes (\jira{RFC-249})? How does it relate to other
  dataset packages, and to making test data available to developers?}

  \item{By what mechanism, if any, can users ``drill down'' from SQuaSH to
  detailed analysis of processing results? Which tools will be provided within
  the drill-down environment to help?}

\end{itemize}

\section{Organization}

\subsection{Meetings and Activities}

The WG will have a scheduled weekly meeting. Other meetings may be called by
the chair on an ad hoc basis.

Members are expected to reserve several hours per week for WG activities.

\subsection{Reporting}

The WG chair will report on WG activities to the DM Project Manger weekly.

At the conclusion of the WG (\S\ref{sec:period}), a brief summary report and
the collection of uses cases and requirements documentation will be presented
to the DM Leadership Team for acceptance.

\subsection{Membership (Proposed)}
\label{sec:members}

Core members of the WG are as follows:

\begin{itemize}

  \item{John Swinbank (Chair; Alert Production \& System Management)}
  \item{Eric Bellm (Alert Production)}
  \item{Hsin-Fang Chiang (LSST Data Facility)}
  \item{Angelo Fausti (SQuaRE)}
  \item{Simon Krughoff (SQuaRE)}
  \item{Lauren MacArthur (Data Release Production)}
  \item{Tim Morton (Data Release Production)}
  \item{A TBD delegate from the SUIT team}

\end{itemize}

In addition, subject matter experts may be invited to participate in certain
WG activities or to present material to WG meetings.

\section{References}
\label{sec:refs}

\renewcommand{\refname}{}
\bibliography{lsst,lsst-dm,refs,books,refs_ads}

\end{document}
